\documentclass[a4paper,12pt,]{article}
\usepackage[utf8]{inputenc}

\title{Feature 3: Teamübergreifende Produktivitätsanalyse}
\author{A.~Heckl, A.~Kohles, A.~Lehene, A.~Mütter}
\date{Februar 2020}

\usepackage[left=2.5cm,right=2.5cm,top=2.5cm,bottom=2.5cm]{geometry}
\usepackage{natbib}
\usepackage{graphicx}
\usepackage[ngerman]{babel}
\newcommand{\ToDo}[1]{
\begin{center}
\fbox{
\begin{minipage}{0.9\textwidth}
#1
\end{minipage}
}
\end{center}
}

\begin{document}

\maketitle
\begin{figure}
\begin{minipage}[b]{\textwidth}
\ToDo{\tiny\textbf{Hinweis:} Wir waren uns bei der Formulierung des Tagebuchs und des Abschlussberichts über die Richtlinien der gendergerechten Sprache und der gendergerechten Formulierung bewusst.
Aus Gründen der Übersichtlichkeit und der einfacheren Lesbarkeit haben wir uns entschieden darauf hinzuweisen, dass in diesem Dokument mit allen männlichen Formulierungen von Personen (Angestellter, Mitarbeiter, Vorgesetzter) selbstverständlich auch die weiblichen Pendants gemeint sind, und diese in keiner Weise vernachlässigbar oder weniger bedeutungsvoll sind.}
\end{minipage}
\end{figure}

\section{Deskriptive Systembeschreibung}

Das hier betrachtete Szenario ist in einem mittleren bis großen Unternehmen anzusiedeln, bei dem die Mitarbeiter in Teams organisiert sind, jedoch - im Gegensatz zu kleinen Betrieben - dem Management langjährige Erfahrungswerte bezüglich des Verhaltens der Mitarbeiter fehlen.

Gleichzeitig unterliegt die Produktivität eines einzelnen Teams starken Schwankungen. Das deutet darauf hin, dass das Produktivitätsverhalten der Mitarbeiter nicht zeitlich konstant ist, sondern von einer Vielzahl möglicher Einflüsse abhängt. Offensichtlich hat die Arbeitgeberseite, also die Unternehmensführung und das mittlere Management, großes Interesse, die Abhängigkeit einer erhöhten oder erniedrigten Produktivität von äußeren Gegebenheiten genauer zu verstehen, lassen sich doch die Ergbnisse einer derartigen Analyse direkt für eine langfristige Steigerung der Produktivität nutzen.

Durch die vermehrte Nutzung von Entwicklertools wie JIRA lassen sich wertvolle Einblicke in das Produktivitätsverhalten eines Mitarbeiters gewinnen. Insbesondere werden in JIRA Issues einem einzelnen Mitarbeiter zugewiesen, der fortan für dessen Lösung verantwortlich ist. Daraus lässt sich ein Maß für die Produktivität eines Mitarbeiters ableiten, nämlich die Zahl der pro Zeiteinheit gelösten Issues. Wir machen hier die Annahme, dass dieses Maß die tatsächliche Produktivität eines Mitarbeiters in ausreichender Näherung beschreibt. Eine einfache und offensichtliche Anwendung ist dann, herauszufinden, welcher Mitarbeiter den nach dieser Metrik größten Beitrag zum Gesamterfolg des Teams geleistet hat, nämlich derjenige, der die meisten Issues lösen konnte.

Desweiteren lässt sich mit diesem Produktivitätsmaß dann untersuchen, inwieweit die Produktivität eines Mitarbeiters zeitlich variiert. So ist es einerseits möglich, dass die Produktivität innerhalb eines einzelnen Tages schwankt, sie könnte beispielsweise morgens und direkt nach der Mittagspause niedriger sein als am späten Vormittag oder am mittleren Nachmittag, wo konzentriertes Arbeiten möglich ist. Andererseits ist es interessant zu untersuchen, ob, und wenn ja wie, die Produktivität innerhalb einer Arbeitswoche schwankt. Sie könnte zum Beispiel an einem Tag gegen Ende der Woche durch die Vorfreude auf das kommende Wochenende höher sein als zu Beginn der Woche, wo die Stimmung durch die sprichwörtlich bekannte ``Montagslaune'' gedrückt ist. Darüber hinaus sind auch noch längere Zeiträume denkbar, von jahreszeitlichen Schwankungen bis hin zu Änderungen innerhalb ganzer Jahre, die jeweils auf Faktoren wie vorweihnachtliche Stressituationen oder die allgemeine Urlaubszeit zurückzuführen sind.

Bisher bezog sich die Analyse stets auf einen Mitarbeiter als Individuum und dessen zeitlich variierende Produktivitätsrate. Bei einem im Team agierenden Mitarbeiter kann man jedoch noch einen Schritt weiter gehen und sich fragen, ob und welchen Einfluss die (ihrerseits ja zeitlich variable) Zusammensetzung des Teams auf die Produktivität eines bestimmten Mitarbeiters hat, beziehungsweise ob das Team als Ganzes bestimmte Vorlieben teilt. Auf diese Weise gewonnene Erkenntnisse lassen sich direkt nutzen: leidet beispielsweise die Produktivität eines Mitarbeiters über die Maßen unter der Anwesenheit eines bestimmten anderen, könnte ein direkter Weg diese Erkenntnis zum Nutzen einer Produktivitätssteigerung sein, die Home-Office Tage der beiden Mitarbeiter so abzustimmen, dass ein möglichst geringer Überlapp an Anwesenheitszeiten am Arbeitsplatz resultiert. Genauso kann bei Präferenzen, die alle Teammitglieder teilen (z.B.~Montag als unproduktivster Tag) entsprechend darauf eingegangen werden.

All diesen Vorschlägen, die schlussendlich alle den einen Zweck erfüllen, die Produktivität zu steigern, stehen zurecht Bedenken seitens der Mitarbeiter und deren Vertretung gegenüber. Inbesondere gehen alle bisherigen Vorschläge mit einem Minimum an Überwachung und daher einer möglichen Verletzung der Privatsphäre einher, genauso wie die aus dem Ergebnis der Analyse gezogenen Konsequenzen die Autonomie der Mitarbeiter (im Beispiel die Wahl der Home-Office Tage) in Mitleidenschaft ziehen können.


\section{Wertekonflikte}

Die Fragen, die es zu beantworten gilt, sofern der Einsatz eines wie oben beschriebenen Systems gegen ethische Bedenken abgesichert sein soll, lassen sich grob in zwei Problemkreise einteilen.

Einerseits stellt sich die Frage, ob das Nachverfolgen und Auswerten persönlicher Daten, die die eigene Leistungsfähigkeit abbilden, in diesem Ausmaß zulässig ist, zumal dies natürlich auch eine mittelfristige Speicherung der betreffenden Daten erfordert. Weiterhin muss untersucht werden, welche Folgen die so gewonnenen Erkenntnisse im Arbeitsleben eines Mitarbeiters nach sich ziehen dürfen, ohne dessen Autonomie zu gefährden. Diese beiden Bereiche übergreifend gilt es zu beachten, dass eine Abhängigkeit der Produktivität eines Mitarbeiters von der An- oder Abwesenheit eines anderen oft Folge privater Beziehungen gleich welcher Art zwischen den Mitarbeitern ist. Deswegen muss man sich überlegen, ob es vertretbar ist, derart tief ins Privatleben der Mitarbeiter einzudringen.

An dieser Stelle empfiehlt es sich, die für die Konzeption der Produktivitätsanalyse gemachten Prämissen genauer zu betrachten. Wir machen die Annahme, dass jedem Mitarbeiter ein gewisses Maß an Autonomie als Grundrecht zusteht. Stillschweigend haben wir genauso vorausgesetzt, dass eine gesteigerte Produktivität sofort als positiv für die Gesamtentwicklung des Unternehmens angesehen werden darf und der Effekt der Maßnahme nr positiv, aber in keinem Fall (zumindest aus Sicht des Unternehmens) negativ ausfallen kann. Langfristige Effekte (Ermüdung der Mitarbeiter durch hohe Arbeitsbelastung) werden ausgeblendet. Dieser Aspekt ist durchaus interessant, da ein genaue Analyse schnell dazu führt, dass sich die Mitarbeiter dem ohnehin vorhandenen konstanten Leistungsdruck noch mehr ausgesetzt fühlen. Dem ist entgegenzusetzen, dass die Auswertung grundsätzlich durch die Aggregation der Daten aller Mitarbeiter im Team kommt und die Angriffsfläche des einzelnen Mitarbeiters reduziert wird. Ebenso muss überprüft werden, inwiefern die Anzahl der abgearbeiteten Issues tatsächlich eine gute Näherung für die Produktivitätsrate eines Mitarbeiters ist. Selbst wenn dies der Fall ist, liefert die Analyse der so gesammelten Daten nur Korrelationen zu äußeren Einflüssen, die nicht zwangsläufig kausal mit einer veränderten Produktivitätsrate zusammenhängen müssen. Schlussendlich steht immer die Frage im Raum, ob die vielleicht tatsächlich vorhandenen Zusammenhänge zwischen äußeren Umständen und der Produktivität nicht durch Geschehnisse im Privatleben der Mitarbeiter überformt sind, auf die der Unternehmer keinen Einfluss hat (und auch nicht haben sollte). Wie bei jedem System, das einen bestimmten Zweck verfolgt, muss auch hier überprüft werden, ob das System nicht durch seinen Einsatz, sei es durch Designfehler oder absichtlichen Missbrauch, zu Ergebnissen führt, die nicht im Sinne des eigentlichen Zwecks stehen. So sollte bestenfalls augeschlossen werden, dass die Resultate der Analyse in die Beziehung der Mitarbeiter untereinander eingreift. Genauso hat der Unternehmer ein Interesse daran zu verhindern, dass die Mitarbeiter ihre Analysewerte durch geeignete Aktionen absichtlich verfälschen, um bestimmte eigene Ziele zu erreichen.

Zusammenfassend ergibt sich das Bild, dass der Implementierung eines Produktivitätsratendokumentationssystems zunächst nur eine Anzahl lösbar erscheinender Probleme entgegensteht, und sich die eigentlichen Probleme im Einsatz der gewonnenen Erkenntnisse verbergen.


\section{Ethische Systemüberprüfung}

Hier soll das Für und Wider aus deontologischer und konsequentialistischer Sichtweise diskutiert werden.

\subsection{Deontologische Betrachtung}

Beginnen wir hier mit den Argumenten, die für eine Implementierung und den Einsatz eines solchen Systems sprechen. Sicherlich ist anzunehmen, dass jeder Arbeitnehmer ein gewisses Interesse daran hat, dass seine Arbeitszeit nicht vergeudet sondern so sinnvoll wie möglich eingesetzt wird. Das beschriebene System ist ein Schritt in genau diese Richtung. Setzt man voraus, dass ein Arbeitgeber die Verpflichtung hat, mit der Arbeitskraft seiner Mitarbeiter so sorgfältig wie nur irgend möglich umzugehen, könnte man sogar noch weitergehen und sich auf den Standpunkt stellen, dass das Unterlassen solcher Optimierungen, obwohl die technischen Mittel zur Verfügung stehen, verwerflich ist, da hierdurch wertvolle Lebenszeit der Mitarbeiter unwiederbringlich verloren geht. Somit ist es durchaus im Interesse eines jeden Mitarbeiters, dass er davon ausgehen kann, an seinem Arbeitsplatz ein soweit möglich optimiertes Arbeitsumfeld vorzufinden. Ein ähnliches Argument gilt auch für die Selbstreflexion der Mitarbeiter: so kann man annehmen, dass jeder Mitarbeiter ein intrinsisches Recht (und Interesse) daran hat herauszufinden, wo seine Stärken und Schwächen liegen. Im Prinzip ist diese Selbstreflexion die fundamentalste Form der Autonomie, denn das Wissen wo die eigenen Enwicklungspotentiale liegen ist Voraussetzung dafür, tatsäclich an sich arbeiten zu können. Den Arbeitnehmern diese Einsicht zu verwehren stellt in dieser Argumentationsweise ein großes Versäumnis dar.

Allerdings gibt es aus deontologischer Sicht auch Argumente gegen das Tracking der Produktivität und deren Optimierung. In jedem Team gibt es einzelne Mitarbeiter, die selbst wissen, dass sie bezüglich ihrer Leistungsfähigkeit mit dem restlichen Team nicht mithalten können und dies auch nie (trotz aller Bemühungen um optimierte Verhältnisse) tun werden. Diese Mitarbeiter wissen genau, dass sie sich glücklich schätzen können, es überhaupt an ihre momentane Position geschafft zu haben, weswegen die Einführung einer rein datenbasierten Produktivitätsanalyse für sie zwangsläufig einen harten Einschnitt darstellen wird, den sie mit Sicherheit nicht befürworten können. Darüber hinaus muss man sich fragen, ob es - selbst für den leistungsstärksten Mitarbeiter - vertretbar ist, den Wert eines Arbeitnehmers für sein Unternehmen rein auf seine Leistungsfähigkeit zu reduzieren. In Anbetracht der Tatsache, dass es um Menschen geht, die einen Großteil ihres Lebens am Arbeitsplatz verbringen werden, und eben nicht um Maschinen, deren einziger Daseinszweck die Erfüllung einer bestimmten Aufgabe ist, muss diese Frage sicherlich mit Nein beantwortet werden. Schlussendlich muss man bedenken, dass - auch wenn aus der Datenanalyse keine direkten Konsequenzen für die Mitarbeiter resultieren - allein die Tatsache, dass die Produktivität getrackt wird, die Mitarbeiter dazu animiert werden, sich hauptsählich um ihre persönliche Produktivitätsrate zu kümmern und nicht mehr um ihre Kollegen. Das Setzen derartiger Akzente kann nicht im unbeschränkten Maße vertretbar sein.


\subsection{Konsequentialistische Betrachtung}

Auch aus konsequentialistischer Sicht sprechen einige Punkte gegen die Einführung eines solchen Systems. Als erstes ist hier anzumerken, dass besagte Zusammenhänge zwischen der Anwesenheit der Mitarbeiter und der Produktivität ihrer Kollegen schnell das Resultat persönlicher Beziehungen ist. Deshalb führt die Optimierung der Anwesenheitszeiten in diesen Fällen möglicherweise zu einem Eingriff in das Privatleben der Mitarbeiter, eine Situation, die weder im Sinne des Mitarbeiters noch des Unternehmers sein kann, da einerseits der Mitarbeiter sicherlich unglücklich über den besagten Eingriff ist, und andererseits unzufriedene Mitarbeiter typischerweise schlechtere Arbeit leisten. Hand in Hand damit geht die Sorge, dass die Mitarbeiter sich viel zu sehr auf die getrackte Produktivitätsrate konzentrieren und so von der Arbeit abgelenkt werden, was natürlich das zugrundeliegende Ziel der Produktivitätssteigerung konterkariert. Selbst wenn dieser Aspekt ausgeschlossen werden kann, bleibt noch immer die Frage im Raum, ob der ständige Produktivitätsdruck zu psychischen Schäden führt, wodurch langfristig auch wieder der gegenteilige Effekt erzielt wird als gewollt. Langfristig muss auch befürchtet werden, dass das Image des Unternehmes nach außen wie nach innen Schaden nimmt, wodurch nicht zuletzt der ein oder andere Mitarbeiter das Unternehmen verlassen wird. Vor allem zu Zeiten von Fachkräftemangel ist dies für das Unternehmen schnell sehr schädlich. Auch muss bedacht werde, dass die Erhebung und Auswertung der Daten, sowie das danach folgende Umsetzen der Erkenntnisse keinen geringen Arbeitsaufwand für typischerweise wertvolle (und damit teure) Mitarbeiter, nämlich Softwareentwickler und Manager bedeutet. Dieser Aufwand muss im Nachgang erst einmal durch ein Mehr an Produktivität ausgeglichen werden, was im Prinzip dazu führt, dass die Mitarbeiter in Summe mehr belastet werden, auch wenn sie effizienter arbeiten, was wiederum zu den oben beschriebenen Konsequenzen führt (psychische Belastung, Verlassen des Unternehmens).

Allerdings existieren auch konsequentialistische Argumente, die für den Einsatz einer Produktivitätsratenkontrolle sprechen. Allem voran führt die gesteigerte Effizienz - richtig umgesetzt - dazu, dass die Mitarbeiter ein festes Arbeitspensum in weniger Zeit leisten können, wodurch sie mehr Freizeit haben, was dem Durchschnittsarbeitnehmer sicherlich zusagen wird. Selbst in der Arbeit sind die Mitarbeiter zufriedener, weil die gesteigerte Effizienz Aufgaben leichter erscheinen lässt, und so die allgemeine Stimmung verbessert wird. Insbesondere wird hier die psychologische Eigenart des Menschen ausgenutzt, solche Erfolge vornehmlich als eigene Leistung zu werten, sodass im Team jeder im Glauben leben darf, die gesteigerte Produktivität sei auf seine Optimierung zurückzuführen. Dies ist auch der Tatsache geschuldet, dass der hier beschriebene Ansatz tendenziell die tatsächlichen Gründe für Produktivitätseinbußen erkennt (Vorliebe von Mitarbeitern für ein bestimmtes Umfeld), statt nur Symptome zu bekämpfen (z.B.~Einführung des nächsten Developertools). Schlussendlich ist natürlich bei gesteigerter Produktivität der erhöhte Gewinn des Unternehmens zu nennen, der entweder den Arbeitnehmern in Form von Freizeitausgleich (s.~oben) oder höherem Gehalt, oder aber der Allgemeinheit durch Spenden oder Stiftungen zugute kommen kann.


\section{Theoretische Deliberation und Urteilsphase}

Die Frage die sich angesichts dieser Überlegungen stellt ist, ob wir es wollen, dass eine solche Technologie in unserer Arbeitswelt Einzug hält und somit die Überwachung wieder ein Stück ausgeweitet wird. Alles in allem scheint es so, dass der Implementierung eines Systems wie beschrieben nichts Grundsätzliches im Wege steht, sofern gewisse Kompromisse eingegangen werden. Diese Kompromisse werden notwendig, weil jede personenbezogene Auswertung, die Mitarbeiter untereinander zu intim vergleicht, deontologisch abzulehnen ist. Zu groß wäre der Eingriff in die Autonomie und Privatsphäre des Mitarbeiters. Gleiches gilt dementprechend auch dafür, welche Maßnahmen in Folge der Auswertung ergriffen werden. Diese dürfen in keinem Fall einer einzelnen Person gelten oder auf deren Verhalten zurückzuführen sein, da solche individuellen Maßnahmen nicht nur deontologisch, sondern auch vom konsequentialistischen Standpunkt abzulehnen sind. Wir erachten es für vertretbar, die Rohdaten der Analyse, sprich die einzelnen Abarbeitungsvorgänge mit Datum, in zusammengefasster Form zugänglich zu machen, da diese Daten im Prinzip für jeden in JIRA angemeldeten Mitarbeiter leicht zugänglich sind: Es gibt eine überschaubare Anzahl von Vorgängen pro Tag, und so könnte ein interessierter Mitarbeiter ohne erheblichen Aufwand (sogar mit Stift und Papier) diese Daten für sich und alle anderen Kollegen erheben. So bleiben unserer Ansicht nach zwei Wege die Daten zu sammeln und zu nutzen: Zum einen halten wir es für zulässig jedem einzelnen Mitarbeiter die Rohdaten seiner eigenen Produktivität zur Verfügung zu stellen, nämlich zu welchen Zeiten (z.B. Uhrzeit oder Wochentag) er wie produktiv war. Hierbei handelt es sich um Daten, die der Arbeitnehmer genauso selbst hätte sammeln können, er wird folglich nur seitens des Arbeitgebers unterstützt. Dieselben Informationen bekommt der Teamleiter für alle Mitarbeiter seiner Abteilung. Was der Mitarbeiter under der Teamleiter mit den Rohdaten anfangen, und welche Schlüsse und Konsequenzen sie daraus ziehen, bleibt ihnen selbst überlassen. Insbesondere wird darauf verzichtet, die Produktivitätsrate mit der Anwesenheit einzelner Mitarbeiter in Beziehung zu setzen, was im Gegensatz zu den Rohdaten eine Information wäre, die nicht jeder JIRA User schon selbst extrahieren kann. Die zweite Möglichkeit sehen wir darin, die Daten im Durchschnitt für alle Mitarbeiter auszuwerten, und auf dieser Basis Maßnahmen zu ergreifen, die wiederum für alle Mitarbeiter gleich gelten. Dadurch wird verhindert, dass einzelne Mitarbeiter im Fokus solcher Maßnahmen stehen. Solche Maßnahmen - oft in Form von Angeboten - gibt es bereits in einer Vielzahl von realen Fällen, das in vielen Unternehmen vorhandene Angebot von Home-Office dürfte hier beispielhaft sein. Dadurch, dass die Maßnahmen für alle gleich gelten ist den Mitarbeitern zudem leichter die Möglichkeit gegeben, gegen diese Maßnahmen vorzugehen, falls sie damit unzufrieden sind, als dies bei individuellen Maßnahmen der Fall ist.


\section{Technische Umsetzung}

Die in Abschnitt 4 dargestellten Kompromisse wurden weitgehend unverändert in die Implementierung übernommen. Um die Implementierung zu testen muss ein Projekt in JIRA vorhanden sein.

Die in JIRA vorhandenen Daten werden dann benutzt um die Produktivität der Mitarbeiter über die gesamte Zeitdauer zu analysieren. Hierzu werden alle Issues benutzt, deren Status ``Closed'' ist, sprich die abgeschlossen sind. Da JIRA das Datum, an dem der Status vergeben wurde nicht direkt abspeichert, wurde das Datum der letzten Änderung (``updated'') benutzt. Jeder geschlossene Issue besitzt im Regelfall einen zuständigen Mitarbeiter, den Assignee. Es wird davon ausgegangen, dass dieser Mitarbeiter die Hauptlast der Arbeit an dem betreffenden Issue zu tragen hat, deswegen wird der Abschluss des Issues dem Assignee als Arbeitsleistung zugeschrieben. Falls (wie in wenigen Einzelfällen) kein Assignee angegeben ist, wird der Issue bei der Auswertung außenvorgelassen. Konkret werden vier Zeitreihen pro Mitarbeiter erstellt: (i) Eine für die Variation der Produktivität in Abhängigkeit der Tageszeit nach Stunden von 7 Uhr morgens bis 21 Uhr abends, (ii) für die Variation während der Woche (Montag bis Freitag), (iii) nach Monaten und (iv) nach Jahreszeit. Dazu werden alle importierten Issues aus der REST API gefetcht und der Reihe nach durchgegangen. Für jeden Issue werden gemäß dem letzten update-Datum dann Eintragungen in den vier Zeitreihen des jeweiligen Assignees vorgenommen. So erfolgt beispielsweise bei einem Issue, den der Benutzer ``asoldano'' am 11.11.2020 um 11:11 Uhr geschlossen hat, in den Zeitreihen dieses Benutzers Eintragungen bei ``11 Uhr'' (Tageszeit), ``Montag'' (Wochentag), ``November'' (Monat) und ``Herbst'' (Jahreszeit). 

Beispielhaft wurde in unserem Plugin die Ansicht für den Teamleiter implementiert. Er kann hier mittels eines Drop-down Menüs einen Mitarbeiter und eine der vier Zeitreihen auswählen, die dann als Histogramm angezeigt wird. Um die Leistung des ausgewählten Mitarbeiters einordnen zu können wird noch bei jeder Größe der Firmenschnitt mit angezeigt. Außer den Rohdaten bekommt der Teamleiter angezeigt, welche der Variablen (Tageszeit, Wochentag, Monat, Jahreszeit) den größten Einfluss auf die Produktivität des gesamten Teams hat. Insbesodere wurde (aus den oben erläuterten Gründen) darauf verzichtet, diese Analyse auf dem Level einzelner Mitarbeiter zu machen. Der Einfluss einer Variablen auf die Produktivität wird wiefolgt gemessen: pro Mitarbeiter wird die Entropie der zu der jeweiligen Variablen gehörigen Zeitreihe bestimmt, sodass man pro Mitarbeiter vier Entropie-Werte bekommt. Jeder dieser vier Einträge wird dann über alle Mitarbeiter gemittelt. Ein niedrigerer Wert der Entropie deutet auf eine weniger gleichmäßige Verteilung hin, was von uns so interpretiert wird, dass eine stärkere Abhängigkeit der Verteilung von der zugrundeliegenden Variablen besteht. Würde zum Beispiel ein Mitarbeiter jeden Monat immer die selbe Anzahl an Issues lösen, wäre der Wert der Entropie seiner monatlichen Verteilung maximal, die Abhängigkeit der Produktivität vom betrachteten Monat jedoch offensichtlich minimal. In unserem Plugin wird die Variable mit der niedrigsten Entropie - sprich der stärksten Abhängigkeit bestimmt und das Ergebnis dem Teamleiter per Infobox angezeigt.


Auf technischer Ebene wurde mit Hilfe des Plugins Charts.js in Javascript das oben beschriebene Diagramm erzeugt.
Nach dem Import des Plugins, sowie dem Anlegen des Charts in einem Canvas im HTML Part, kann dieses dynamisch aus dem Javascript File befüllt werden.
Im Diagramm werden per Balkendiagramm verschiedene Kennzahlen des Mitarbeiters mit dem Durchschnitt der Firma verglichen. Hierbei kann die Produktivität im Verlauf der Tageszeit, des Wochentags, der Monate oder der Jahrezeit angezeigt werden. Die Auswahl hierfür wird mittels zwei Drop Down Menüs vorgenommen (eines für die Auswahl des Mitarbeiters und eines für die Auswahl des Zeitraums). Nach jeder Auswahl in einem Drop Down wird das Diagramm dynamisch aktualisiert (mittels ``{\tt myChart.update()}'') und die entsprechenden Kennzahlen angezeigt.
Die vorher angezeigten Daten müssen aus dem dataset-Array des Charts entfernt werden (mittels {\tt pop()}). Die neu anzuzeigenden Daten in das dataset-array des Charts gepusht werden. Die Überschrift des Diagramms wird ebenfalls bei jeder Neuauswahl mit aktualisiert.
Die Drop-Down-Menüs werden zu Programmstart initialisiert und mit den entsprechenden Mitarbeiter IDs aus der bereitgestellten Datei bestückt oder den zur Verfügung stehenden Kennzahlen bestückt. Der aktuell ausgewählte Mitarbeiter und die aktuell ausgewählte anzuzeigende Kennzahl wird per globaler Variable gesichert.

Die weiteren verwendeten Optionen der Diagramme (Padding, Anzeigen von Labels und Legenden, Schrittweiten der Diagramme, \dots) sind dem Programmcode zu entnehmen.
Wichtig ist anzumerken, dass der Parameter ``responsive'' auf ``false'' gesetzt wurde, sodass die Diagramme beim Zoomen im Browser nicht mit skalieren, um eine sinnvolle Anzeige auf allen Displaygrößen zu gewährleisten (da bei sehr kleinen Bildschirmen Teile des Diagramms ohne Zoom nicht zu lesen waren).

Die notwendigen Änderungen im Style wurden direkt im gadget.xml file vorgenommen, um den Transfer der Dateien übersichtlich und kompakt zu halten. Eine Einbindung in den CSS Ordner hätte eine zusäzliche Änderung im atlassian-plugin.xml file bedingt. Daher wurde es der Einfachheit halber in den HTML Teil integriert.

\paragraph{Testen des Systems.} Um die Implementierung zu testen wurden beispielhaft Daten aus dem WildFly-Projekt von RedHat\footnote{Für das Projekt selbst siehe {\tt https://wildfly.org/}, und für den Issuetracker {\tt https://issues.redhat.com/projects/WFLY/issues}.} importiert. Die Daten enthalten die \emph{geschlossenen} Issues des Projektes über eine Zeitspanne mehrerer Monate. Mittels dieser Daten wurde die Funktionalität unseres Plugins getestet und etwaige Mängel ausgeräumt.

\end{document}
